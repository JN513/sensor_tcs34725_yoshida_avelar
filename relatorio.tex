\documentclass[12pt,a4paper]{article}
\usepackage[utf8]{inputenc}
\usepackage[brazil]{babel}
\usepackage{graphicx}
\usepackage{hyperref}
\usepackage{amsmath}
\usepackage{geometry}
\geometry{margin=2.5cm}
\setlength{\parskip}{6pt}
\setlength{\parindent}{0pt}

% ---------------------------------------------------------------
% Cabeçalho e informações
% ---------------------------------------------------------------
\title{Relatório Técnico — Sensor de Cor GY-33 (TCS34725)}
\author{
Ana Beatriz Barbosa Yoshida — RA: 245609 — \texttt{@beatrizbarbosay} \\
Julio Nunes Avelar — RA: 241163 — \texttt{@JN513} \\
Turma EA701 — 2025S2
}
\date{11 de Novembro de 2025}

\begin{document}
\maketitle

\begin{center}
\textbf{Repositório:} \url{https://github.com/JN513/sensor_tcs34725_yoshida_avelar}
\end{center}

% ---------------------------------------------------------------
\section{Escopo e Objetivos}
O presente experimento teve como objetivo desenvolver um sistema embarcado capaz de detectar e identificar cores utilizando o sensor óptico \textbf{GY-33 (TCS34725)} acoplado à placa \textbf{BitDogLab}. 
O projeto foi implementado em linguagem \textbf{C/C++} utilizando o \textbf{Pico SDK}, explorando os recursos de comunicação digital \textbf{I2C} e controle de brilho via \textbf{PWM por hardware} do microcontrolador \textbf{RP2040}.

Os principais objetivos técnicos foram:
\begin{itemize}
    \item Implementar comunicação I2C com o sensor TCS34725;
    \item Calibrar e ler os canais de cor (\texttt{R}, \texttt{G}, \texttt{B} e \texttt{C});
    \item Converter os valores normalizados para os formatos RGB (0–255) e HSV;
    \item Controlar o LED RGB da BitDogLab via PWM;
    \item Classificar a cor dominante de acordo com faixas de matiz (Hue);
    \item Reproduzir a cor detectada no LED RGB, garantindo coerência visual.
\end{itemize}

O sistema foi considerado funcional quando a cor reproduzida pelo LED correspondia à cor observada no objeto e a classificação textual coincidiu com o esperado.

% ---------------------------------------------------------------
\section{Metodologia e Implementação}

\subsection{Arquitetura do Sistema}
O sistema é composto por dois módulos principais:

\begin{enumerate}
    \item \textbf{Sensor de cor (GY-33 / TCS34725)} — responsável pela leitura das intensidades de luz refletida nos canais vermelho, verde, azul e claro, comunicando-se com o microcontrolador via I2C.
    \item \textbf{LED RGB da BitDogLab} — reproduz a cor detectada por meio de três canais PWM independentes (R, G e B).
\end{enumerate}

\begin{center}
\begin{verbatim}
+------------------+       +----------------+       +-----------------+
|    TCS34725      | <---> |   BitDogLab    | --->  | LED RGB PWM     |
|  Sensor de cor   |  I2C  |  RP2040 MCU    |  RGB  | Saídas R,G,B    |
+------------------+       +----------------+       +-----------------+
\end{verbatim}
\end{center}

\subsection{Ligações Elétricas}
\begin{center}
\begin{tabular}{|c|c|c|c|}
\hline
\textbf{Pino Sensor} & \textbf{GPIO BitDogLab} & \textbf{Função} & \textbf{Observação} \\
\hline
VCC & 3.3V & Alimentação & Compatível 3.3V–5V \\
GND & GND & Terra comum & — \\
SDA & GPIO0 & I2C0 SDA & Dados \\
SCL & GPIO1 & I2C0 SCL & Clock \\
\hline
\end{tabular}
\end{center}

O LED RGB integrado à BitDogLab utiliza os pinos:
\begin{center}
R = GPIO13, \quad G = GPIO11, \quad B = GPIO12
\end{center}
Ajustes de brilho foram necessários para evitar mistura excessiva de branco.

\subsection{Desenvolvimento de Software}
O firmware foi desenvolvido utilizando o \textbf{Pico SDK v2.2.0}, com as seguintes bibliotecas:
\begin{itemize}
    \item \texttt{hardware\_i2c.h} — comunicação com o sensor;
    \item \texttt{hardware\_pwm.h} — controle do LED RGB;
    \item \texttt{pico/stdlib.h} e \texttt{stdio.h} — depuração e temporização.
\end{itemize}

\textbf{Fluxo lógico do programa:}
\begin{enumerate}
    \item Inicializa periféricos (UART, I2C e PWM);
    \item Lê valores dos canais \texttt{R}, \texttt{G}, \texttt{B} e \texttt{C};
    \item Normaliza os dados com base no canal “clear”;
    \item Converte os valores RGB para HSV;
    \item Classifica a cor dominante de acordo com o matiz (Hue);
    \item Atualiza o LED RGB via PWM;
    \item Exibe no terminal a cor detectada e seu matiz.
\end{enumerate}

\subsection{Conversão RGB-->HSV e Classificação de Cores}
Foi implementado um algoritmo de conversão RGB-->HSV para detecção perceptiva de cores.  
A classificação foi baseada em faixas de matiz (Hue):

\begin{center}
\begin{tabular}{|c|c|}
\hline
\textbf{Faixa de Hue (°)} & \textbf{Cor Dominante} \\
\hline
0–15 ou $>$345 & Vermelho \\
15–45 & Laranja \\
45–70 & Amarelo \\
70–160 & Verde \\
160–200 & Ciano \\
200–260 & Azul \\
260–320 & Roxo \\
320–345 & Magenta \\
\hline
\end{tabular}
\end{center}

As faixas foram ajustadas empiricamente para melhorar a distinção entre tons intermediários.

\subsection{Controle PWM do LED RGB}
Cada canal foi configurado com frequência de 1 kHz e resolução de 16 bits.  
Os valores normalizados (0–255) são convertidos em \emph{duty cycle} proporcional, invertido devido ao LED de ânodo comum.  
Foi realizada calibração manual para equilibrar o brilho entre canais.

% ---------------------------------------------------------------
\section{Resultados e Análise}
Os testes foram conduzidos sob iluminação branca ambiente.  
O sistema foi capaz de identificar e reproduzir as cores principais com boa correspondência entre leitura e exibição.

\begin{center}
\begin{tabular}{|c|c|c|c|}
\hline
\textbf{Cor Observada} & \textbf{Hue (°)} & \textbf{Cor Detectada} & \textbf{RGB Reproduzido} \\
\hline
Garrafinha vermelha & $\sim$5 & Vermelho & (255, 0, 0) \\
Camisa verde & $\sim$130 & Verde & (0, 255, 0) \\
Camisa azul & $\sim$230 & Azul & (0, 0, 255) \\
Garrafinha amarela & $\sim$55 & Amarelo & (255, 255, 0) \\
Mesa branca & — & Branco/Cinza & (255, 255, 255) \\
\hline
\end{tabular}
\end{center}

\textbf{Observações:}
\begin{itemize}
    \item Em tons claros, o LED tende ao branco ou cinza;
    \item A detecção de preto exigiu limiar mínimo no canal “clear”.
\end{itemize}

A figura \ref{fig:montagem} mostra a montagem do sistema, enquanto a figura \ref{fig:saida} apresenta a saída no terminal durante os testes.

\begin{figure}[htbp]
\centering
\includegraphics[width=1\textwidth]{imgs/montagem.png}
\caption{Montagem do sistema na BitDogLab.}
\label{fig:montagem}
\end{figure}


\begin{figure}[htbp]
\centering
\includegraphics[width=1\textwidth]{imgs/saida.png}
\caption{Saída no terminal durante o teste.}
\label{fig:saida}
\end{figure}

% ---------------------------------------------------------------
\section{Dificuldades e Soluções}
\begin{itemize}
    \item \textbf{Brilho excessivo no LED RGB:} calibração de brilho para evitar saturação branca.
    \item \textbf{Limites imprecisos entre cores próximas:} ampliação das faixas de Hue.
    \item \textbf{Influência da iluminação ambiente:} padronização da luz de teste.
    \item \textbf{Sensibilidade desigual dos canais RGB:} compensação manual de ganho (redução no canal verde).
\end{itemize}

% ---------------------------------------------------------------
\section{Conclusões e Trabalhos Futuros}
O projeto alcançou os objetivos propostos, demonstrando o funcionamento correto do sensor TCS34725 na detecção de cores.  
A integração entre leitura I2C, conversão HSV e controle PWM resultou em um sistema eficiente, didático e de boa responsividade.

\textbf{Trabalhos futuros sugeridos:}
\begin{itemize}
    \item Implementar calibração automática com referência branca;
    \item Adicionar compensação de luminosidade ambiente (auto gain);
    \item Integrar exibição da cor detectada em display OLED;
    \item Expandir para reconhecimento de padrões multicoloridos;
\end{itemize}

% ---------------------------------------------------------------
\section{Referências}
\begin{itemize}
    \item ams AG. \textit{TCS34725 Color Light-to-Digital Converter with IR Filter — Datasheet.} Disponível em: \url{https://cdn-shop.adafruit.com/datasheets/TCS34725.pdf}
    \item Raspberry Pi Foundation. \textit{Pico SDK — C/C++ Development Guide.} Disponível em: \url{https://www.raspberrypi.com/documentation/microcontrollers/}
    \item BitDogLab. \textit{Documentação de Hardware — Mapeamento de GPIOs e periféricos.}
    \item Adafruit Industries. \textit{TCS34725 Color Sensor Guide.} Disponível em: \url{https://learn.adafruit.com/adafruit-color-sensors}
    \item Repositório do projeto: \url{https://github.com/JN513/sensor_tcs34725_yoshida_avelar}
\end{itemize}

\end{document}
